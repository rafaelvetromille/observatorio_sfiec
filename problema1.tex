\documentclass[11pt]{article}
\usepackage[a4paper,margin=2cm]{geometry}
\usepackage[brazilian]{babel}
\usepackage[utf8]{inputenc}
\usepackage[T1]{fontenc}
\linespread{1.3}
\parskip=12pt
\parindent=0pt
\usepackage{enumitem}
\usepackage{amsmath}
\usepackage{amsfonts, tabularx}
\usepackage{graphicx}
\usepackage{amssymb}
\usepackage{hyperref}
\usepackage{amsthm}
\usepackage{color}
\usepackage{placeins}
\usepackage{caption} 
\captionsetup[table]{skip=10pt}


% Defining the question styles
\theoremstyle{definition}
\newtheorem{prob}{Problema}

% Custom commands
\newcommand{\E}{\mathbb{E}}
\newcommand{\Var}{\mathrm{Var}}
\newcommand{\Prob}{\mathbb{P}}

% declare a new theorem style
\newtheoremstyle{solution}%
{1pt}% Space above
{1pt}% Space below 
{\itshape\color{red}}% Body font
{}% Indent amount
{\bfseries\color{red}}% Theorem head font
{.}% Punctuation after theorem head
{.5em}% Space after theorem head
{}% Theorem head spec (can be left empty, meaning ‘normal’)

\theoremstyle{solution}
\newtheorem*{solution}{Solution}

% --- Code starts here ---
\begin{document}
	\begin{center}
		{\Large{\textbf{Case 1 - Vaga: Pesquisador Credenciado}}}\\
		\vspace{0.2cm}
		Assunto: Análise dos dados da RAIS 2020 para o estado do Ceará - Setor: Indústria Calçadista\\
		Candidato: Rafael Vetromille
	\end{center}
	
\begin{prob}
A análise dos dados da RAIS 2020 para a indústria calçadista cearense nos permite avaliar a dinâmica e importância desse mercado. Primeiramente, podemos avaliar os municípios que a indústria de calçados mais emprega. A Tabela \ref{tab1} mostra esses resultados.

% Table created by stargazer v.5.2.3 by Marek Hlavac, Social Policy Institute. E-mail: marek.hlavac at gmail.com
% Date and time: qua, mai 25, 2022 - 13:54:56
\begin{table}[!htbp] \centering 
  \caption{Número de funcionários, indústria calçadista, por municípios do Ceará}
  \label{tab1} 
\begin{tabular*}{\textwidth}{l @{\extracolsep{\fill}} c}
\hline 
Nome do Município & Quantidade de funcionários no setor \\ 
\hline 
Sobral & 15888 \\ 
Horizonte & 10672 \\ 
Quixeramobim & 5292 \\ 
Morada Nova & 4466 \\ 
Itapipoca & 3719 \\ 
Fortaleza & 2894 \\ 
Santa Quitéria & 2602 \\ 
Crato & 2502 \\ 
Juazeiro Do Norte & 2319 \\ 
Brejo Santo & 2280 \\ 
\hline
\end{tabular*} 
\begin{flushleft}
\footnotesize Fonte: RAIS, 2020
\end{flushleft}
\end{table} 

A Tabela \ref{tab1} representa os 10 principais municípios empregadores da indústria calçadista. O município de Sobral destaca-se como o município com maior número de funcionários nessa indústria. Seguido por Horizonte, Quixeramobim, Morada Nova e os demais municípios. O Gráfico \ref{fig1} a seguir representa a dispersão dos municípios no estado do Ceará.

\begin{figure}[htp!]
\centering
\includegraphics[scale=0.37]{graph1.png}
\caption{Número de funcionários no setor de calçados, por município, CE}
\label{fig1}
\end{figure}%%
\newpage
É possível notar que a maioria dos municípios empregadores da indústria calçadistas são aqueles ao redor de Fortaleza. Além disso, podemos notar também que dos 184 municípios cearenses, 44 são produtores de calçados, isto representa que aproximadamente 24\% dos municípios cearenses são produtores de calçados. A Tabela \ref{tab2} exibe a remuneração média para os dez principais municípios produtores de calçados. O município de Sobral, como sendo o principal empregador da indústria calçadista, também é aquele que tem a maior remuneração média. Ressalta-se a importância de Fortaleza como sendo 
o segundo município com maior remuneração média.

\begin{table}[!htbp] \centering 
  \caption{Remuneração Média, por município, CE} 
  \label{tab2} 
\begin{tabular*}{\textwidth}{l @{\extracolsep{\fill}} c @{\extracolsep{\fill}} r}
\hline
Nome do município & Quantidade de funcionários no setor & Remuneração Média (R\$) \\ 
\hline 
Sobral & 15888 & R\$ 1 188,44 \\ 
Horizonte & 10672 & R\$ 1 181,28 \\ 
Quixeramobim & 5292 & R\$ 928,91 \\ 
Morada Nova & 4466 & R\$ 1 091,14 \\ 
Itapipoca & 3719 & R\$ 1 083,11 \\ 
Fortaleza & 2894 & R\$ 1 186,88 \\ 
Santa Quitéria & 2602 & R\$ 1 115,31 \\ 
Crato & 2502 & R\$ 1 187,86 \\ 
Juazeiro Do Norte & 2319 & R\$ 933,66 \\ 
Brejo Santo & 2280 & R\$ 1 011,50 \\ 
\hline 
\end{tabular*} 
\begin{flushleft}
\footnotesize Fonte: RAIS, 2020
\end{flushleft}
\end{table} \vspace{-0.5cm}

As principais ocupações desse setor podem ser verificadas na Tabela \ref{tab3}. A ocupação dominante, com quase 30 mil funcionários, é a de trabalhadores polivalentes que são  aqueles trabalhadores que atuam em várias áreas, de maneira rotativa, sem uma tarefa fixa. Em segundo lugar, encontra-se a ocupação de costurador, muito importante nesse ramo.

\begin{table}[!htbp] \centering 
  \caption{Principais ocupações empregadores e remuneração média, CE} 
  \label{tab3} 
\begin{tabular*}{\textwidth}{l @{\extracolsep{\fill}} c @{\extracolsep{\fill}} r}
\hline
Ocupação & Qtd. & Remuneração Média (R\$) \\ 
\hline 
Trabalhador polivalente da confecção de calçados & 29979 & R\$ 963,72 \\ 
Costurador de calçados, a  máquina & 6926 & R\$ 984,25 \\ 
Preparador de calçados & 6856 & R\$ 882,16 \\ 
Alimentador de linha de produção & 6717 & R\$ 1 001,40 \\ 
Sapateiro (calçados sob medida) & 3606 & R\$ 1 047,40 \\ 
Moldador de plástico por injeção & 2669 & R\$ 960,84 \\ 
Supervisor  (indústria de calçados e artefatos de couro) & 1441 & R\$ 2 818,51 \\ 
Almoxarife & 1307 & R\$ 1 157,33 \\ 
Montador de calçados & 1081 & R\$ 873,21 \\ 
Acabador de calçados & 1036 & R\$ 1 003,82 \\ 
\hline
\end{tabular*} 
\begin{flushleft}
\footnotesize Fonte: RAIS, 2020
\end{flushleft}
\end{table}

\newpage

Em relação à divisão entre trabalhadores homens ou mulheres, a indústria calçadista do Ceará emprega mais homens do que mulheres, com os homens representando, aproximadamente, 55\% ($\sim$40 mil) da mão de obra total desse setor. O Gráfico \ref{fig2} representa essa dinâmica em mais detalhes. A remuneração média dos homens nesse setor foi de R\$ 1182,00 enquanto que a das mulheres foi de R\$ 1011,00 (RAIS, 2020).
\begin{figure}[htp!]
\centering
\includegraphics[scale=0.35]{graph2.png}
\caption{Quantidade de funcionários, por sexo, setor de calçadista, CE}
\label{fig2}
\end{figure}

Por fim, podemos analisar a dispersão dos funcionários do setor de calçados do Ceará sob a ótimo do grau de instrução. A maioria dos funcionários da indústria calçadista cearense possuem ensino médio completo ($\sim$62\%) e possuem uma remuneração média de R\$ 1043,45. A Tabela \ref{tab4} apresenta esse resultado para os demais graus de instrução.

\begin{table}[!htbp] \centering 
  \caption{Grau de instrução e Remuneração média} 
  \label{tab4} 
\begin{tabular*}{\textwidth}{l @{\extracolsep{\fill}} r @{\extracolsep{\fill}} r}
\hline 
Grau de Instrução & Qtd. & Remuneração Média (R\$) \\ 
\hline \\[-1.8ex] 
Ensino Médio Completo & 45094 & R\$ 1 043,45 \\ 
Ensimo Médio Incompleto & 8027 & R\$ 1 001,00 \\ 
Fundamental Completo & 6509 & R\$ 1 012,26 \\ 
De 6º ao 9º do Ensino Fundamental & 4871 & R\$ 967,36 \\ 
Até 5º ano Incompleto & 2210 & R\$ 1 051,92 \\ 
Ensino Superior Incompleto & 2184 & R\$ 1 616,72 \\ 
Ensio Superior Completo & 1748 & R\$ 3 452,72 \\ 
Até o 5º Completo & 1237 & R\$ 1 023,20 \\ 
Analfabeto & 314 & R\$ 1 001,49 \\ 
Mestrado & 4 & R\$ 8 598,76 \\ 
Doutorado & 2 & R\$ 8 542,44 \\ 
Total (Ceará) & 72200 & R\$ 1 062,20 \\ 
\hline
\end{tabular*} 
\begin{flushleft}
\footnotesize Fonte: RAIS, 2020
\end{flushleft}
\end{table} 

\end{prob}

\end{document}